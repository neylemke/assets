
\documentclass[twoside]{article}
\usepackage[affil-it]{authblk}
\usepackage{lipsum} % Package to generate dummy text throughout this template
\usepackage{eurosym}
\usepackage[sc]{mathpazo} % Use the Palatino font
\usepackage[T1]{fontenc} % Use 8-bit encoding that has 256 glyphs
\usepackage[utf8]{inputenc}
\linespread{1.05} % Line spacing-Palatino needs more space between lines
\usepackage{microtype} % Slightly tweak font spacing for aesthetics

\usepackage[hmarginratio=1:1,top=32mm,columnsep=20pt]{geometry} % Document margins
\usepackage{multicol} % Used for the two-column layout of the document
\usepackage[hang,small,labelfont=bf,up,textfont=it,up]{caption} % Custom captions under//above floats in tables or figures
\usepackage{booktabs} % Horizontal rules in tables
\usepackage{float} % Required for tables and figures in the multi-column environment-they need to be placed in specific locations with the[H] (e.g. \begin{table}[H])
\usepackage{hyperref} % For hyperlinks in the PDF

\usepackage{lettrine} % The lettrine is the first enlarged letter at the beginning of the text
\usepackage{paralist} % Used for the compactitem environment which makes bullet points with less space between them

\usepackage{abstract} % Allows abstract customization
\renewcommand{\abstractnamefont}{\normalfont\bfseries} 
%\renewcommand{\abstracttextfont}{\normalfont\small\itshape} % Set the abstract itself to small italic text

\usepackage{titlesec} % Allows customization of titles
\renewcommand\thesection{\Roman{section}} % Roman numerals for the sections
\renewcommand\thesubsection{\Roman{subsection}} % Roman numerals for subsections
\titleformat{\section}[block]{\large\scshape\centering}{\thesection.}{1em}{} % Change the look of the section titles
\titleformat{\subsection}[block]{\large}{\thesubsection.}{1em}{} % Change the look of the section titles

\usepackage{fancyhdr} % Headers and footers
\pagestyle{fancy} % All pages have headers and footers
\fancyhead{} % Blank out the default header
\fancyfoot{} % Blank out the default footer
\fancyhead[C]{X-meeting $\bullet$ November 2017 $\bullet$ S\~ao Pedro} % Custom header text
\fancyfoot[RO,LE]{} % Custom footer text

%----------------------------------------------------------------------------------------
% TITLE SECTION
%----------------------------------------------------------------------------------------

\title{\vspace{-15mm}\fontsize{24pt}{10pt}\selectfont\textbf{IN SILICO IDENTIFICATION, CHARACTERIZATION AND PHYLOGENETIC ANALYSIS OF miRNAs IN WILD PEPPER}} % Article title

\author{Ailton Pereira da Costa Filho$^1$, Monize Angela de Andrade$^1$, Laurence Rodrigues do Amaral$^1$, Matheus de Souza Gomes$^1$}

\affil{1 UFU\\ }
\vspace{-5mm}
\date{}

%----------------------------------------------------------------------------------------

\begin{document}

\maketitle % Insert title

\thispagestyle{fancy} % All pages have headers and footers

%----------------------------------------------------------------------------------------
% ABSTRACT
%----------------------------------------------------------------------------------------

\begin{abstract}
Capsicum annuum var. Glabriusculum is a species of wild pepper with perennial and woody growth that due to its organoleptic characteristics is used in food as a flavoring. It is an important source of germplasm for the Capsicum genus, especially when used as a source of genes for resistance to disease. Due to human invasion, inadequate harvests and environmental degradation, their survival is threatened. Recently, the plant transcriptome has received attention from the scientific community to identify which miRNAs are regulating gene expression. The miRNAs are a class of small non-coding RNAs which length ranges from 20 to 24 nucleotides, and perform regulatory function in the organism. This class of small RNAs is involved in several biological functions, such as cell proliferation, apoptosis, and stress response. The objective of this work was to identify, characterize and analyze phylogenetically, putative miRNAs of Capsicum annuum var. glabriusculum and their orthologs. We searched for the probable mature miRNAs and precursors using miRBase. Pre-miRNAs were characterized as to their structural and thermodynamic characteristics. The conservation and alignment analyzes were performed using ClustalX 2.1 and RNAalifold. The secondary structures of the pre-miRNAs were obtained by RNAfold. Phylogenetic analysis of C. annuum var. glabriusculum pre-miRNAs and their orthologs using MEGA v5.2 was also performed. From the analysis, 101 putative miRNA families were obtained, and the families MIR-160, MIR-162, MIR-164, MIR-390, MIR-393 and MIR-828 showed high conservation in Solanaceae. It is worth mentioning that the targets of these families were described. The phylogenetic analysis of these miRNA families showed high conservation within their families and the phylogenetic distribution corroborated with the plant life tree. When comparing the secondary structures of the orthologs with the precursors it was evident that the pre-miRNAs are also conserved, mainly within the family Solanaceae. Thus, the obtained results amplify the understanding of the miRNA pathway in wild Capsicum annuum var. glabriusculum opening space for new inquiries regarding the regulation of gene expression in this species.

Funding: FAPEMIG, CNPq, UFU and CAPES
\end{abstract}
\end{document}